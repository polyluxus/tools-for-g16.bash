\documentclass[   % inherits article
  final,          % or
  % draft,
  a4paper,        % or
  % letterpaper   % (etc. pp.)
  rscols=3,       % this is the default
  margin=1.0cm,   % default, or
]{refsheet}

% Typesetting codeprettier
\usepackage[dvipsnames]{xcolor}
\definecolor{lightgray}{gray}{0.95}
\usepackage{listings}
\lstdefinestyle{mytex}{
  language=[LaTeX]TeX,%
  backgroundcolor=\color{lightgray},%
  basicstyle=\footnotesize\ttfamily,%
  keywordstyle=\bfseries,%
  breaklines=true,%
  morekeywords={part,chapter,subsection,subsubsection,paragraph,subparagraph}%
}

\lstset{style=mytex}

% Symbols for SE and GH
\usepackage{fontawesome} 
% Symbols for creative commons
\usepackage{ccicons}
% linking to the outside world
\usepackage{hyperref}
% Martin's weird SE handle
% (use a different package and Xe/LuaLaTeX if you're doing proper Japanese cheat sheets.)
\usepackage{CJKutf8} 

% Please no font errors
% https://tex.stackexchange.com/a/9366/33413
\newcommand\mybackslash{\char`\\}

% What this is about
\title{Cheat-Sheat for \texttt{tools-for-g16.bash} (0.0.14, 2018-08-20)}
\author{Martin C Schwarzer}
\date{\today}

\begin{document}
\maketitle

\section{Introduction}

This accompanies the repository \href{https://github.com/polyluxus/tools-for-g16.bash}{polyluxus/tools-for-g16.bash}.

Various bash scripts to aid the use of the quantum chemistry software package Gaussian 16.

\section{\texttt{g16.prepare.sh}}
\section{\texttt{g16.testroute.sh}}
\section{\texttt{g16.freqinput.sh}}
\section{\texttt{g16.submit.sh}}
\section{\texttt{g16.getenergy.sh}}
\section{\texttt{g16.getfreq.sh}}
\section{\texttt{g16.chk2xyz.sh}}

\begin{rslisttt}{-m <ARG>}
  \item[-m <ARG>] Memory
\end{rslisttt}

\section{Author, Bugs, and the Rest}

Martin C Schwarzer  
\faStackExchange~\href{https://chemistry.stackexchange.com/users/4945}{Martin-\begin{CJK*}{UTF8}{min}マーチン\end{CJK*}}
\faGithub~\href{https://github.com/polyluxus}{polyluxus}

This document is licensed \ccbysa.

\end{document}
