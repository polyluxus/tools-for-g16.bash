\documentclass{standalone}

\usepackage{tikz}
\usetikzlibrary{matrix,calc,shapes}
\usepackage{amssymb}

\begin{document}
%define styles for nodes (from http://latex-cookbook.net/articles/flowchart/)
\tikzset{
  treenode/.style = {shape=rectangle, rounded corners,
                     draw, anchor=center,
                     text width=5cm, align=center, font=\rmfamily\normalsize,
                     top color=white, bottom color=blue!20,
                     inner sep=1ex},
  root/.style     = {treenode, font=\Large,
                     bottom color=red!30},
  decision/.style = {treenode, diamond, inner sep=0pt, aspect=2},
  env/.style      = {treenode, font=\ttfamily},
  archive/.style  = {env, bottom color=orange!40},
  result/.style   = {treenode, bottom color=green!40},
  finish/.style   = {root, bottom color=green!40},
  dummy/.style    = {circle,draw}
}

%\tikzset{
%  treenode/.style = {shape=rectangle, rounded corners,
%                     draw, anchor=center,
%                     text width=5cm, align=center, font=\rmfamily\normalsize,
%                     top color=white, bottom color=blue!20,
%                     inner sep=1ex},
%  root/.style     = {treenode, font=\Large,
%                     bottom color=red!30},
%  decision/.style = {treenode, diamond, inner sep=0pt, aspect=2},
%  env/.style      = {treenode, font=\ttfamily},
%  archive/.style  = {env, bottom color=orange!40},
%  result/.style   = {treenode, bottom color=green!40},
%  finish/.style   = {root, bottom color=green!40},
%  dummy/.style    = {circle,draw}
%}


\begin{tikzpicture}[-latex]
  \matrix (chart)
    [
      matrix of nodes,
      column sep      = 1cm,
      row sep         = 1cm,
      column 1/.style = {nodes={archive}},
      column 2/.style = {nodes={env}},
      column 3/.style = {nodes={env}},
      column 4/.style = {nodes={env}},
      column 5/.style = {nodes={archive}}
    ]
    {
      %1
      & & %
      |[root]| Optimised TS Structure          & \\
      %2
      & & %
      |[archive]| <method>.freq.com, <method>.freq.gjf, <method>.freq.log, <method>.freq.chk, \dots & & %
      |[result]| \(E_\mathrm{ZPE}, E_\mathrm{o}, \dots, G\)\\
      %3
      &
      <method>.irc.fwd.com & &%
      <method>.irc.rev.com  & \\
      %4
      <method>.irc.fwd.log, <method>.irc.fwd.chk, (queue error/out) &
      <method>.irc.fwd.gjf, <method>.irc.fwd.<queue>.bash      & & %
      <method>.irc.rev.gjf, <method>.irc.rev.<queue>.bash      & 
      <method>.irc.rev.log, <method>.irc.rev.chk, (queue error/out) & \\
      %5
      (<method>.irc.fwd.xyz,) <method>.irc.fwd.fchk & & & & %
      (<method>.irc.rev.xyz,) <method>.irc.rev.fchk & \\
      %6
      & 
      <method>.irc.fwd.opt.com & &
      <method>.irc.fwd.opt.com & \\
      %7
      <method>.irc.fwd.opt.log, <method>.irc.fwd.opt.chk, (queue error/out) &
      <method>.irc.fwd.opt.gjf, <method>.irc.fwd.opt.<queue>.bash      & & %
      <method>.irc.rev.opt.gjf, <method>.irc.rev.opt.<queue>.bash      & 
      <method>.irc.rev.opt.log, <method>.irc.rev.opt.chk, (queue error/out) & \\
      %8
      <method>.irc.fwd.opt.xyz, <method>.irc.fwd.opt.fchk & %
      |[decision]| Stat. Point? & & %
      |[decision]| Stat. Point? & %
      <method>.irc.rev.opt.xyz, <method>.irc.rev.opt.fchk & \\
      %9
      & 
      <method>.irc.fwd.opt.freq.com & &
      <method>.irc.fwd.opt.freq.com & \\
      %10
      <method>.irc.fwd.opt.freq.log, <method>.irc.fwd.opt.freq.chk, (queue error/out) & %
      <method>.irc.fwd.opt.freq.gjf, <method>.irc.fwd.opt.freq.<queue>.bash & & % 
      <method>.irc.rev.opt.freq.gjf, <method>.irc.rev.opt.freq.<queue>.bash & % 
      <method>.irc.rev.opt.freq.log, <method>.irc.rev.opt.freq.chk, (queue error/out) \\
      %11
      <method>.irc.fwd.opt.freq.xyz, <method>.irc.fwd.opt.freq.fchk & 
      |[decision]| NImag? & & %
      |[decision]| NImag? & %
      <method>.irc.rev.opt.freq.xyz, <method>.irc.rev.opt.freq.fchk & \\
      %12
      |[result]| \(E_\mathrm{ZPE}, E_\mathrm{o}, \dots, G\) & & %
      |[finish]| Done! & & 
      |[result]| \(E_\mathrm{ZPE}, E_\mathrm{o}, \dots, G\) & \\
    };
  \draw
  (chart-1-3) edge node [right] {\dots} (chart-2-3);
  \draw
  (chart-2-3) edge node [above] {\texttt{g16.getfreq}} (chart-2-5);
  \draw
  (chart-2-3) |- node [near start, right] {\texttt{g16.ircinput}} (chart-3-2) ;
  \draw
  (chart-2-3) |- (chart-3-4) ;
  \draw
  (chart-3-2) edge node [right] {\texttt{g16.submit}} (chart-4-2);
  \draw
  (chart-3-4) edge node [right] {\texttt{g16.submit}} (chart-4-4);
  \draw
  (chart-4-2) edge node [above] {\texttt{g16}} (chart-4-1);
  \draw
  (chart-4-4) edge node [above] {\texttt{g16}} (chart-4-5);
  \draw
  (chart-4-2) edge node [right] {\texttt{g16.optinput}} (chart-6-2);
  \draw
  (chart-4-4) edge node [right] {\texttt{g16.optinput}} (chart-6-4);
  \draw
  (chart-4-1) edge node [right] {\texttt{g16.chk2xyz}} (chart-5-1);
  \draw
  (chart-4-5) edge node [right] {\texttt{g16.chk2xyz}} (chart-5-5);
  \draw
  (chart-4-1) -- +(0,-1) -| node [pos=0.3, below] {inspect \texttt{*.log}} (chart-6-2);
  \draw
  (chart-4-5) -- +(0,-1) -| node [pos=0.3, below] {inspect \texttt{*.log}} (chart-6-4);
  \draw
  (chart-6-2) edge node [right] {\texttt{g16.submit}} (chart-7-2);
  \draw
  (chart-6-4) edge node [right] {\texttt{g16.submit}} (chart-7-4);
  \draw
  (chart-7-2) edge node [above] {\texttt{g16}} (chart-7-1);
  \draw
  (chart-7-4) edge node [above] {\texttt{g16}} (chart-7-5);
  \draw
  (chart-7-1) edge node [right] {\texttt{g16.chk2xyz}} (chart-8-1);
  \draw
  (chart-7-5) edge node [right] {\texttt{g16.chk2xyz}} (chart-8-5);
  \draw
  (chart-7-2) edge node [right] {inspect \texttt{*.log}} (chart-8-2);
  \draw
  (chart-7-4) edge node [right] {inspect \texttt{*.log}} (chart-8-4);
  \draw
  (chart-8-2) -- +(+4,0) node [near start, above] {\textbf{no}} |- ($(chart-7-2)+(0,+3)$) ;
  \draw
  (chart-8-4) -- +(-4,0) node [near start, above] {\textbf{no}} |- ($(chart-7-4)+(0,+3)$) ;
  \draw
  (chart-8-2) edge node [left] {\textbf{yes}} node [right] {\texttt{g16.freqinput}} (chart-9-2) ;
  \draw
  (chart-8-4) edge node [left] {\textbf{yes}} node [right] {\texttt{g16.freqinput}} (chart-9-4) ;
  \draw
  (chart-9-2) edge node [right] {\texttt{g16.submit}} (chart-10-2);
  \draw
  (chart-9-4) edge node [right] {\texttt{g16.submit}} (chart-10-4);
  \draw
  (chart-10-2) edge node [above] {\texttt{g16}} (chart-10-1);
  \draw
  (chart-10-4) edge node [above] {\texttt{g16}} (chart-10-5);
  \draw
  (chart-10-1) edge node [right] {\texttt{g16.chk2xyz}} (chart-11-1);
  \draw
  (chart-10-5) edge node [right] {\texttt{g16.chk2xyz}} (chart-11-5);
  \draw
  (chart-10-1) -- +(-3,0) |- (chart-12-1) node [pos=0.3, sloped, below] {\texttt{g16.getfreq}} ;
  \draw
  (chart-10-5) -- +(+3,0) |- (chart-12-5) node [pos=0.3, sloped, above] {\texttt{g16.getfreq}} ;
  \draw
  (chart-10-2) edge node [right] {inspect \texttt{*.log}} (chart-11-2);
  \draw
  (chart-10-4) edge node [right] {inspect \texttt{*.log}} (chart-11-4);
  \draw
  (chart-11-2) |- node [near start, left] {\(0\)} (chart-12-3);
  \draw
  (chart-11-4) |- node [near start, left] {\(0\)} (chart-12-3);
  \draw
  (chart-11-2) -- +(+4,0) node [near start, above] {\(\geq1\)} |- ($(chart-7-2)+(0,3)$);
  \draw
  (chart-11-4) -- +(-4,0) node [near start, above] {\(\geq1\)} |- ($(chart-7-4)+(0,3)$);
%  \draw
%  (chart-7-1) edge node [above] {\texttt{g16}} (chart-7-2);
%  \draw
%  (chart-7-2) -| (chart-8-3) node [near end, right] {\texttt{g16.getenergy}} ;
%  \draw
%  (chart-9-1) edge node [above] {\(0\)} (chart-9-2);
%  \draw
%  (chart-9-2) edge (chart-9-3);
%  \draw
%  (chart-9-1) |- (chart-10-2) node [near start, right] {\(1\)} ;
%  \draw
%  (chart-10-2) edge (chart-10-3);
%  \draw
%  (chart-9-1) -- +(-3,0) |- (chart-2-1)
%  node [near start, sloped, above] {\(\geq2\); extract molecular structure, repeat optimisation};
%  \draw
%  (chart-9-3) -- +(+4,0) node [near start, above] {\textbf{no}}
%  |- (chart-2-1) node [near start, sloped, below] {extract molecular structure, repeat optimisation};
%  \draw
%  (chart-10-3) -- +(0,-2) node [near start, right] {\textbf{no}}
%  -| ($(chart-2-1)+(-4,0)$) node [near start, sloped, above] {extract molecular structure, modify/scan, repeat optimisation} -- (chart-2-1);
%  \draw
%  (chart-9-3) -- +(0,-2) node [near start, right] {\textbf{yes}} -| (chart-11-4);
%  \draw
%  (chart-10-3) -| (chart-11-4) node [near start, sloped, above] {\textbf{yes}}; 
\end{tikzpicture}

\end{document}
