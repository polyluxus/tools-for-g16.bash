\documentclass[   % 
  final,          % 
  a4paper         % 
]{article}

\usepackage[a4paper,margin=2cm]{geometry}
\setcounter{secnumdepth}{0}
\usepackage{hyperref}
\usepackage[inline]{enumitem}
\usepackage{calc}
% Intermediately include a new list type to import the cheat sheet
\newenvironment{rslisttt}[1]
{
  \begin{description}[labelwidth=\widthof{\texttt{#1}},font=\ttfamily]
}{
  \end{description}
}

\begin{document}
\section{Introduction}

This accompanies the repository \href{https://github.com/polyluxus/tools-for-g16.bash}{polyluxus/tools-for-g16.bash}.

Various bash scripts to aid the use of the quantum chemistry software package Gaussian 16.

\subsection{Preliminary notes}

The notation in brackets \texttt{[ ]} indicate optional arguments/inputs;
arguments in angles \texttt{< >} require human input;
a bar \texttt{|} indicates alternatives.

The following abbreviations will be used:
\begin{rslisttt}{ARG}
  \item[opt] Short for option(s)
  \item[ARG] String type argument
  \item[INT] Positive integer (including zero)
  \item[NUM] Whole number (including zero)
  \item[FLT] Floating point number
  \item[DUR] Duration in format \texttt{[[HH:]MM:]SS} 
\end{rslisttt}
\end{document}
