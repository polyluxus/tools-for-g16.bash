\documentclass[   % 
  final,          % 
  a4paper         % 
]{article}

\usepackage[a4paper,margin=2cm]{geometry}
\setcounter{secnumdepth}{0}

\usepackage{hyperref}
\usepackage{listings}

\begin{document}
\section{Installation \& Configuration}
\label{sec:installation}

The scripts of this repository are not self-contained,
they each need access to the resources directory 
and it has to be called like that.

The easiest way to install the script is to clone the git repository from GitHub:\newline
\href{https://github.com/polyluxus/tools-for-g16.bash}{github.com/polyluxus/tools-for-g16.bash}.
Alternatively you can download the tarball archive of the latest release from there, too:\newline
\href{https://github.com/polyluxus/tools-for-g16.bash/releases/latest}{github.com/polyluxus/tools-for-g16.bash/releases/latest}.

After unpacking it only needs to be configured.
There are a two kinds of file names recognised:
\lstinline`g16.tools.rc` and the hidden \lstinline`.g16.toolsrc`.
The repository comes with an example of the former, 
therefore updating the repository will (probably) overwrite the file.
The latter file is excluded from this process,
and therefore has always precedence, 
so it is generally a safer option to configure.

The scripts will search for these configuration settings in the following order of directories:
\begin{enumerate}
  \item the path to the script itself 
  \item the user's home directory
  \item the \texttt{.config} directory in the user's home directory
  \item the current working directory, i.e. from where the script is called.
\end{enumerate}
Only the last found file will be applied.

The repository comes with a configuration script in the \texttt{configure} directory.
It produces a formatted file like \lstinline`g16.tools.rc` from old or the default settings.
While you can assign an arbitrary filename, 
I recommend to store as \lstinline`.g16.toolsrc` in the root directory of the repository.

To make the scripts accessible from anywhere, the directory where they have been stored
must be found in the \texttt{PATH} variable.
Alternatively, you can create softlinks to them in a directory, 
which is already recognised by \texttt{PATH}.
A common setting for this is the local \texttt{\$HOME/bin}.
The configure script will, if desired, try to create these softlinks (omitting the \texttt{sh} suffix).

\subsection{Dependencies}

For displaying the help files, the command line utility \texttt{grep} is required.

Some scripts need access to an istallation of Open Babel;\footnote{%
  For more information see \href{http://openbabel.org/}{http://openbabel.org/}.}
it is mainly used to interface to the TurboMole \texttt{coord} file format,
and the post-processing scripts.

It should be no surprise that an installation of Gaussian 16 is necessary for
some of the scripts, currently used are the utilities \texttt{testrt} and \texttt{formchk}.
While a main wrapper (see next chapter) is still not included in the repository,
the scripts will accept these.
Otherwise the Gaussian environment has to be set up, 
and the utilities must be found in the \texttt{PATH} directories.
The direct path to these utilities might work, too.

The following example is a very simple wrapper script when using software modules. 
\begin{lstlisting}[language=bash]
#!/bin/bash
module load gaussian
"$@"
\end{lstlisting}
For the RWTH clusters, which have such an environment installed, 
there is a separate repository with such wrapper scripts:
\href{https://github.com/polyluxus/rwth-tools}{github.com/polyluxus/rwth-tools}.

\end{document}
