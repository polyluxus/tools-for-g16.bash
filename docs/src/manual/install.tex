\documentclass[   % 
  final,          % 
  a4paper         % 
]{article}

\usepackage[a4paper,margin=2cm]{geometry}
\setcounter{secnumdepth}{0}

\usepackage{hyperref}
\usepackage{listings}

\begin{document}
\section{Installation \& Configuration}

The scripts of this repository are not self-contained,
they each need access to the resources directory 
and it has to be called like that.

The easiest way to install the script is to clone the git repository from GitHub:\newline
\href{https://github.com/polyluxus/tools-for-g16.bash}{github.com/polyluxus/tools-for-g16.bash}.
Alternatively you can download the tarball archive of the latest release from there, too:\newline
\href{https://github.com/polyluxus/tools-for-g16.bash/releases/latest}{github.com/polyluxus/tools-for-g16.bash/releases/latest}.

After unpacking it only needs to be configured.
There are a two kinds of file names recognised:
\lstinline`g16.tools.rc` and the hidden \lstinline`.g16.toolsrc`.
The repository comes with an example of the former, 
therefore updating the repository will (probably) overwrite the file.
The latter file is excluded from this process,
and therefore has always precedence, 
so it is generally a safer option to configure.

The scripts will search for these configuration settings in the following order of directories:
\begin{enumerate}
  \item the path to the script itself 
  \item the user's home directory
  \item the \texttt{.config} directory in the user's home directory
  \item the current working directory, i.e. from where the script is called.
\end{enumerate}
Only the last found file will be applied.

The repository comes with a configuration script in the \texttt{configure} directory.
It produces a formatted file like \lstinline`g16.tools.rc` from old or the default settings.
While you can assign an arbitrary filename, 
I recommend to store as \lstinline`.g16.toolsrc` in the root directory of the repository.

To make the scripts accessible from anywhere, the directory where they have been stored
must be found in the \texttt{PATH} variable.
Alternatively, you can create softlinks to them in a directory, 
which is already recognised by \texttt{PATH}.
A common setting for this is the local \texttt{\$HOME/bin}.
The configure script will, if desired, try to create these softlinks (omitting the \texttt{sh} suffix).

\end{document}
